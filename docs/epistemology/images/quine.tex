\documentclass[12pt,letterpaper]{report}
\usepackage{graphicx}

\setlength{\parindent}{2.5em}
\setlength{\oddsidemargin}{-.25in}
\setlength{\evensidemargin}{-.25in}
\setlength{\footskip}{.5in}
\setlength{\textwidth}{7in}
\setlength{\textheight}{9.5in}
%\setlength{\headheight}{0in} defaults to 0
%\setlength{\topskip}{0in} defualts to 0
%\setlength{\headsep}{0in} defaults to 0
\setlength{\topmargin}{0in}

\begin{document}

\begin{flushright}
Shawn O'Neil\\
PL370\\
3-5 Page 3\\
Fall 2004
\end{flushright}

\section*{\begin{center}Quine's Naturalization and Kim's Response\end{center}}
\renewcommand{\baselinestretch}{2}
\normalsize
\hspace{\parindent}
In W. V. Quine's 1969 article, ``Epistemology Naturalized,'' the author argues for the replacement of ``traditional'' epistemology with one which has been ``naturalized.'' He bases this argument on the failure of other types of epistemology. While Quine would like to argue that all of epistemology prior to his conception of it has failed, he focuses on the dominant forms of epistemology which are foundational in nature. In particular, these forms include the Cartesian and the Carnapian programs. Quine bases his replacement on the fact that these programs fail to reach, and must fail to reach, their respective goals. In this paper I intend to explain the goals of these ``traditional'' programs that Quine attacks as well as show how Quine's program is different. Also I will discuss Jaegwon Kim's 1988 response to Quine --- namely Kim's argument that epistemology is inherently normative, and that Quine's new program fails to be epistemology all together.

In ``Epistemology Naturalized,'' Quine attacks two programs of epistemology which have certain traits in common. The first program is that of the Cartesian. Quine breaks the Cartesian program into two parts, the ``conceptual'' part and the ``doctrinal'' part. Leaving this confusing notation aside, the ``conceptual''  side of the the Cartesian program amounts to the normative (prescriptive) idea of what knowledge (or more precisely justification) is. In other words, the Cartesian seeks to find a set of rules which can be used to determine whether a given belief is justified or not, by reducing the belief in question to foundational beliefs.  The second part, doctrine, attempts to show that a given belief is in fact justified, because it does meet the condition that it can be deduced from some foundational justified beliefs.

Quine's argument relies upon the fact that the Cartesian program fails in its goal of finding anything of substance which is in fact justified. This is mostly due to a lack of foundationally justified beliefs, and the strength of justification required for inference. In fact, almost no one will contend with this. Another type of foundational program attacked by Quine is that of Carnap. The Carnapian program is very similar to that of the Cartesian, in that there is a prescriptive element present which seeks to find out which beliefs are justified by reducing them to notions of sensory terms. Also present is the doctrinal part which attempts to use the prescriptive part to justify beliefs about truths of nature (science), which are reduced to sensory terms. It is similarly agreed that this project fails, both in terms of reduction (not everything can be reduced to sensory terms) and in terms of justifying truths of nature. This last type of failure was shown by Hume in his argument against induction.

In order to look more carefully at Quine's ``replacement'' for epistemology, it would be good to answer a question about the epistemological programs Quine attacks: What would epistemology have gained should the Cartesian or the Carnapian program have proved successful? Both of these programs seek to determine what is justified in an absolute sense --- the sense that a belief is justified if it is indubitable. Also, both of these programs are prescriptive in nature; their success would lead directly to a type of formula which could be applied to any belief to determine if it is justified or not. Thus, not only would justification (and so knowledge) be nailed down as a solid concept, it would be known what knowledge philosophers (and everyone else) in fact have. So far as epistemology goes, this is the holy grail.

What does it seem the failure of these programs amounts to? It certainly shows that neither the Cartesian nor the Carnapian program can lead to a prescriptive idea of justification. Also, neither program can give us that holy grail of epistemology, telling us just what beliefs we have are justified. Due to the nature of the arguments against them, it also seems that any sort of foundational program may be doomed to failure in providing any substantial result whatsoever. Given that these types of foundational programs  have dominated epistemology as a study for nearly it's entire life, this is an extremely important result.

With this in mind, I will now turn to the conception of epistemology which Quine argues should replace ``traditional'' epistemology. Quine argues that because these types of traditional epistemology have failed in their goals concerning justification, epistemology itself should be replaced with a ``naturalized'' version. For Quine, the term ``naturalized'' is akin to the term ``scientific.'' Quine, in effect, would like to turn epistemology into a natural science --- complete with hypothesis, observations, and conclusions. In naturalized epistemology, the subject of the hypotheses, observations, and conclusions is the human being himself. Instead of looking at a person's beliefs and asking ``are those beliefs justified,'' Quine looks at the person and asks simply ``what caused those beliefs.''

The differences between naturalized epistemology and traditional epistemology are indeed staggering. While both traditional and naturalized epistemology look at the relationship between input (sense data) and output (beliefs) in a person, only traditional epistemology looks at the relationship in an evidential or justificatory light. Quine's naturalized epistemology on the other hand, looks at the relationship in a purely causal light. Quine freely admits that this form of epistemology amounts to little more than scientific psychology. Naturalized epistemology throws out any sense of justification of beliefs, focusing on merely the environmental causes of such beliefs. As such, it lacks the normative or prescriptive element found in traditional epistemology. Naturalized epistemology does not address what the rules are for justification, or whether any belief is justified.

Kim's response to the radical view of naturalized epistemology is more or less common sense. Kim argues that the field of epistemology must be essentially normative; it must prescribe conditions for justification (or knowledge) and thereby determine which beliefs are justified (or can be considered knowledge.) To some extent all of the epistemological programs to date have dealt with justification of some sort. The word epistemology itself is defined as the study of the nature of knowledge --- in practice epistemology is the study of a validation, or justification, of a body of knowledge claims. Note that validation is a less loaded word in this context, and perhaps is clearer. Epistemology is the study of justification of knowledge claims, not necessarily a study which defines knowledge in terms of justification for the knower (which is the case for the Cartesian and Carnapian programs.)

Thus, Quine argues for the replacement of traditional epistemology, which has a prescriptive component, with naturalized epistemology, which has no prescriptive or normative component. So, according to Kim, naturalized ``epistemology'' has no bearing on the study of epistemology at all; it is only concerned with causal relationships, not validation of knowledge claims. After all, what rational reason could there be for replacing one method of study with another, which not only has radically different methods of procedure but radically different goals as well?

In the end, it seems Kim's argument against Quine's suggestion of replacing epistemology with a type of psychology holds. While Quine does make good points in his attack on Cartesian and Carnapian programs, these points just do not seem to justify the replacement Quine thinks they do. Nevertheless, because the study of epistemology has been dominated for nearly it's entire lifetime by such reductionist and foundational programs, Quine's article is still hugely impactful. The subsequent movement away from foundational theories to other types such as coherentist and externalist is evidence for this.

\end{document}
