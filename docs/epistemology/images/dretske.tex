\documentclass[12pt,letterpaper]{report}
\usepackage{graphicx}

\setlength{\parindent}{2.5em}
\setlength{\oddsidemargin}{-.25in}
\setlength{\evensidemargin}{-.25in}
\setlength{\footskip}{.5in}
\setlength{\textwidth}{7in}
\setlength{\textheight}{9.5in}
%\setlength{\headheight}{0in} defaults to 0
%\setlength{\topskip}{0in} defualts to 0
%\setlength{\headsep}{0in} defaults to 0
\setlength{\topmargin}{0in}

\begin{document}

\begin{flushright}
Shawn O'Neil\\
PL370\\
3-5 Page 2\\
Fall 2004
\end{flushright}

\section*{\begin{center}Dretske's Epistemic Operators\end{center}}
\renewcommand{\baselinestretch}{2}
\normalsize
\hspace{\parindent}
In 1970 Fred Dretske gave a reply to the classic sceptical argument in a paper entitled ``Epistemic Operators,'' attempting to argue that the sceptic's argument was flawed due to a misunderstanding of the way knowledge works. Dretske's argument attempts to show that epistemic operators do not penetrate to certain consequences of propositions which are relied upon in the classic sceptical argument. Here I will discuss this paper, including what these operators are and what it means to say that they don't penetrate like other operators do. Also, I will explain what it is about the classic sceptical argument that Dretske will attack, and how he goes about this attack via an analogy. Lastly, I am compelled to demonstrate a weakness inherent in Dretske's argument against scepticism.

In ``Epistemic Operators,'' Dretske describes a type of term which is referred to as an ``operator.'' An operator is something that when affixed to a statement operates on it to result in another statement. Examples of some operators include ``it is true that,'' ``it is weird that,'' ``knows that,'' and ``explains that.'' To take a simple example of how an operator operates on a statement, we consider the statement \textit{``the garage is empty.''} Using the operator ``it is true that,'' on this statement, we get \textit{``it is true that the garage is empty.''}

Dretske argues that operators come in three types: non penetrating, fully penetrating, and semi penetrating. The extent to which an operator penetrates is the extent to which it is correct to apply the operator to each side of an implication. A fully penetrating operator, for instance, is an operator which when applied to any given statement, can also be correctly applied to any consequence of that statement. To use a symbolic shorthand, we let $\alpha$ and $\beta$ be statements, $\rightarrow$ be ``implies that,'' and O($\alpha$) be an operated on statement $\alpha$ with the operator O. Thus, for any $\alpha$ and $\beta$, an operator O is fully penetrating when $\alpha$ $\rightarrow$ $\beta$ implies that O($\alpha$) $\rightarrow$ O($\beta$). To help clarify, we will let $\alpha$ be \textit{``Bill and Susan married each other.''} Now, $\beta$ can be anything entailed by $\alpha$, say \textit{``Susan got married.''} Dretske argues that the operator ``it is true that'' is a fully penetrating operator. To test this operator with our $\alpha$ and $\beta$, we look at O($\alpha$) $\rightarrow$ O($\beta$): \textit{``It is true that Bill and Susan married each other implies that it is true that Susan got married.''} The operator works correctly when applied to each side of the implication, thus it penetrates in this case.

Using our same $\alpha$ and $\beta$, we will look at the operator ``it is weird that,'' which Dretske claims is non penetrating. By non penetrating, Dretske doesn't mean that the operator will never penetrate; rather that it will usually not, even to simple logical consequences. The $\alpha$ $\rightarrow$ $\beta$ we have above is such a simple logical consequence: \textit{``Bill and Susan married each other implies that Susan got married.''} However, looking at O($\alpha$) $\rightarrow$ O($\beta$): \textit{``It is weird that Bill and Susan married each other implies it is weird that Susan got married.''} This statement certainly could be doubted; possibly it is weird that Susan and Bill got married (they hated each other) but it probably wouldn't be weird that Susan got married (if she was a nice person.)

The third type of operator is a semi penetrating operator. Easily described, a semi penetrating operator is one which does penetrate to simple logical consequences (is not non penetrating) yet does not penetrate to all consequences (is not fully penetrating.) The main focus of Dretske's paper consists of showing that explanatory  operators (such as ``explains that'') and furthermore epistemic operators (such as ``knows that'') are semi penetrating. Specifically, Dretske attempts to show that the epistemic operators do not penetrate to the types of implications necessary that the classical sceptical argument uses as its basis.

To put a name on the type of penetration sceptics assume that epistemic operators have, I will refer to the ``thesis of penetrability.'' One version of the classic sceptical argument goes like this: 1. If one knows that the world is as it appears, then one also knows that the world is not an illusion. 2. It is not the case that anyone can know that the world is not an illusion. Therefore, by Modus Tollens, it is not the case that anyone knows that world is at it appears. Sceptics rely on the thesis of penetrability in this argument. It is assumed that the implication \textit{``the world is as it appears implies that the world is not an illusion and is caused by an evil genius''} implies that \textit{``one knows that the world is as it seems implies that one knows the world is not an illusion and is caused by an evil genius.''} By challenging the full penetrability of the epistemic operator ``knows that,'' Dretske challenges the thesis of penetrability, in turn challenging the classic sceptical argument.

To challenge the full penetrability of the epistemic operators, Dretske starts by pointing out certain features of the epistemic operators. First Dretske shows that the epistemic operators, particularly ``knows that,'' are not non-penetrating. Using the example above, for the simple logical consequence \textit{``Bill and Susan married each other implies that Susan got married,''} it is correct to say that \textit{``She knows that Bill and Susan married each other implies that she knows that Susan got married.''} In fact, it seems almost trivial to show that ``knows that'' penetrates to simple logical consequences; nevertheless, it is necessary for his greater argument that Dretske points it out.

The second feature Dretske points out about epistemic operators is that they are not fully penetrating. The specific cases in which this is easy to see are when the initial implications are of the type where the consequent is ``presupposed'' in the antecedent. Take an example $\alpha$ $\rightarrow$ $\beta$: \textit{``My brother was not going to move from the seat on the bus implies that it was my brother who was not going to move.''} Using the operator ``knew that,'' it would be incorrect to say \textit{``She knew my brother was not going to move from the seat on the bus implies that she knew that it was my brother who was not going to move.''} Thus, even though the statement $\alpha$ $\rightarrow$ $\beta$ is correct, the operated on statement O($\alpha$) $\rightarrow$ O($\beta$) is incorrect, meaning the operator (in this case "knew that") is not fully penetrating.

Having established these two features of epistemic operators, Dretske moves onto another type of operator which I shall call explanatory operators. The most obvious example of such an operator is ``explains that.'' Dretske argues that the explanatory operators have the same two properties as were described for the epistemic operators, namely that they are not non penetrating (they work for simple logical implications), and that they are not fully penetrating (they do not work for presuppositional implications.)

The same examples as were used with the epistemic operators can be used to show these two properties for the explanatory operators. If $\alpha$ $\rightarrow$ $\beta$ is \textit{``Bill and Susan married each other implies that Susan got married''} then it is correct to say that \textit{``Bill loving Susan explains that Bill and Susan married each other implies that Bill loving Susan explains that Susan got married,''} showing that ``explains that'' is at least not non penetrating. If $\alpha$ $\rightarrow$ $\beta$ is \textit{``My brother was not going to move from the seat on the bus implies that it was my brother who was not going to move''} it is not correct to say that \textit{``There were no seats left on the bus explains that my brother was not going to move from the seat on the bus implies that there were no seats on the bus explains that it was my brother who was not going to move.''} Not only is this statement horribly complex (which is irrelevant) it doesn't make sense to say that there being no seats left explains that it was my brother who wouldn't move. This shows that the explanatory operators are not fully penetrating.

Dretske argues by analogy that any operator type which shares these two features with the explanatory operators (as do the epistemic operators) will probably also share any other features that are inherent to the explanatory operators. He doesn't have just any other feature of explanatory operators in mind here though, he specifically wants to talk about the feature of explanatory operators not penetrating to something called ``contrast consequences,'' which are a type of consequence inherent in the classical sceptical argument.

A contrast consequence is one which is entailed by a statement in a certain way. A contrast consequence of a statement such as \textit{``The wall is red''} is \textit{``The wall is not white and cleverly illuminated to look red.''} To generalize, for any statement that takes the form \textit{A is $\phi$}, the contrast consequence is a statement of the form \textit{A is not ($\rho$ and $\gamma$)} where $\phi$ and $\rho$ are two mutually exclusive properties of A, and $\gamma$ is any property we like. All of this is (according to Dretske) logically entailed by the antecedent A is $\phi$. Notice that the implication \textit{``The world is as it appears implies that it is not an illusion and is caused by an evil genius''} is just such a contrast consequence. 

Dretske shows that explanatory operators such as ``explains that'' do not penetrate to contrast consequences by looking at the example $\alpha$ $\rightarrow$ $\beta$: \textit{``I painted my living room walls green implies that I did not paint my living room walls green and instead bought a new couch cover.''} This example becomes much more clear when we affix the operator ``explains that'' to just $\alpha$ by itself: \textit{``My living room walls and couch clashed explains that I painted my living room walls green.''} But, when we try to affix the operator to both sides of the aforementioned implication, giving O($\alpha$) $\rightarrow$ O($\beta$): \textit{``My living room walls and couch clashed explains that I painted by living room walls implies that my living room walls and couch clashed explains that I did not paint my living room walls green and instead bought a new couch cover,''} we end up with a nonsense statement --- the operator does not penetrate to this contrast consequence. After all, why should the explanation for my painting my walls explain my not just changing my couch cover? Thus, Dretske shows rather strongly that the explanatory operators do not penetrate to his contrast consequences. 

Referring back to his analogy that any operators that share the two features of not non penetrability and not full penetrability should share any other feature, Dretske argues that epistemic operators (which do share these two features) should also share the feature of not penetrating to contrast consequences. Put together with the fact that the sceptic assumes the penetrability of the epistemic operator ``knows that'' in the contrast consequence contained in the first premise of the sceptic's argument (using the thesis of penetrability), Dretske seems to have successfully challenged that first premise.

Nevertheless, as I stated earlier I must demonstrate a weakness inherent to Dretske's argument. I shall be kind, however, and simply discourse for a bit on a weakness that Dretske openly admits in his own paper. As I see it, the main weakness in ``Epistemic Operators'' is the argument using the analogy between explanatory and epistemic operators. As any argument from analogy is going to be inherently weak, a person pursuing this line of argumentation should find as many similarities between the two things being compared as possible for as strong an argument as possible. In Dretske's analogy between explanatory and epistemic operators he only describes two similarities. To many, including myself, this would barely put such similarities outside of the realm of coincidence. Indeed, could it not be simple coincidence that both types were both not non penetrating and not fully penetrating? It would be rather damning to Dretske's argument, were one to find an operator that was not non penetrating and also not fully penetrating, but also was penetrating to contrast consequences. Such a counterexample would show easily that the similarities Dretske found were coincidental.

Fred Dretske's argument against the classic sceptical argument is at once complex and elegant. Also, the argument is inventive, for who else would have thought of challenging something as seemingly obvious as the penetrability of the epistemic operators to certain consequences? Aside from the roundabout manner of doing so by using analogy, Dretske presents a compelling and well thought out argument against scepticism. If nothing else, he has advanced the field of epistemology further through the formation of better and more complete ways of formalizing the language we use every day. If he had contributed nothing else to the field, the concept of epistemic operators would be enough.


\end{document}
