\documentclass[12pt,letterpaper]{article}
\usepackage{graphicx, amssymb, amsmath, amsthm}

\setlength{\parindent}{2.5em}
\setlength{\oddsidemargin}{-.25in}
\setlength{\evensidemargin}{-.25in}
\setlength{\footskip}{.5in}
\setlength{\textwidth}{7in}
\setlength{\textheight}{9.5in}
%\setlength{\headheight}{0in} defaults to 0
%\setlength{\topskip}{0in} defualts to 0
%setlength{\headsep}{0in} defaults to 0
\setlength{\topmargin}{0in}

\newtheorem{defaultThm}{Theorum}
\theoremstyle{definition}
\newtheorem{definition}{Definition}
\newtheoremstyle{mythm}{1em}{1em}{}{1in}{}{}{.5em}{}
\newtheorem*{mythm}{Proof}

\begin{document}

\begin{center}
Shawn O'Neil\\
Math Logic\\
Fall 2004
\end{center}

\section*{\begin{center}Proof: $\sum\cup\{\alpha\}\vdash\phi\Rightarrow\sum\vdash(\alpha\rightarrow\phi)$ (Deduction Theorem)\end{center}}
%\renewcommand{\baselinestretch}{2}
%\normalsize
%\hspace{\parindent}
The Deduction Theorem is a very useful tool in the work of formal logic. However, the Deduction Theorem is a metatheorem, which is to say it is used to deduce the existence of a proof in a given theory from an already existing proof in the given theory, without belonging to the theory itself. First, a few simple definitions and propositions:\vspace{1em}
	\textit{\begin{list}{$\bullet$}{\setlength{\leftmargin}{2in}}
	\item{Def 1: $\Rightarrow$  Implication in metalanguage.}
	\item{Def 2: $\rightarrow$  Implication in object language.}
	\item{Prop 1: $\beta\in\sum\Rightarrow\sum\vdash\beta$}
	\item{Prop 2: $\sum\vdash\gamma$ and $\gamma\rightarrow\alpha\Rightarrow\sum\vdash\alpha$ (Modus Ponens)}
	\item{Prop 3: $\vdash\alpha\rightarrow\alpha$}
	\item{Prop 4: $\vdash\alpha\rightarrow\sum\vdash\alpha$, for any $\sum$}
	\end{list}}\vspace{1em}
Since we have $\sum\cup\{\alpha\}\vdash\phi$, we will let $\phi_{1},\phi_{2},...,\phi_{n}$ be a proof of $\phi$ from $\sum\cup\{\alpha\}$, where $\phi_{n}=\phi$. We will prove by induction on $i$ that $\sum\vdash(\alpha\rightarrow\phi_{i})$. First, notice that $\phi_{1}$ must be in 1 of 3 places: \vspace{1em}
	\renewcommand{\arraystretch}{1.5}
	\begin{center}
	\hspace{0em}
	\begin{array}[b]{l@{\hspace{.75cm}}l@{\hspace{1cm}}}
	(a) & \mbox{in }\sum\\
	(b) & \mbox{axiom of PC}\\
	(c) & \alpha\\
	\end{array}
	\end{center}\vspace{1em}
So, we need to show that for each of these three cases and $i=1$, $\sum\vdash(\alpha\rightarrow\phi_{i})$.\vspace{1em}
	\renewcommand{\arraystretch}{1.5}
	\begin{center}
	\hspace{0em}
	\begin{array}[b]{l@{\hspace{.75cm}}l@{\hspace{1cm}:\,}l}
	(a1) & \phi_{1}\rightarrow(\alpha\rightarrow\phi_{1}) & \mbox{PC Axiom 1}\\
	(a2) & \sum\vdash\phi_{1} & \mbox{Prop 2}\\
	(a3) & \sum\vdash(\alpha\rightarrow\phi_{1})\mbox{ \textbf{for case a}} & \mbox{MP, Prop 1}\\
	(b1) & \vdash(\alpha\rightarrow\phi_{1}) & \mbox{MP, PC Axiom}\\
	(b2) & \sum\vdash(\alpha\rightarrow\phi_{1})\mbox{ \textbf{for case b}} & \mbox{Prop 4}\\
	(c1) & \vdash(\alpha\rightarrow\phi_{1})\mbox{ for case c} & \mbox{Prop 3}\\
	(c2) & \sum\vdash(\alpha\rightarrow\phi_{1})\mbox{ \textbf{for case c}} & \mbox{Prop 4}\\
	\end{array}
	\end{center}\vspace{1em}
Thus, for $i=1$, we have shown that $\sum\vdash(\alpha\rightarrow\phi_{i})$. Next comes the induction step. Assume that $\sum\vdash(\alpha\rightarrow\phi_{k})$, for all $k<i$. Thus, the next step we haven't shown in our proof, $\phi_{i}$, could be in one of 4 places:\vspace{1em}
	\renewcommand{\arraystretch}{1.5}
	\begin{center}
	\hspace{0em}
	\begin{array}[b]{l@{\hspace{.75cm}}l@{\hspace{1cm}}}
	(d) & \mbox{in }\sum\\
	(e) & \mbox{axiom of PC}\\
	(f) & \alpha\\
	(g) & \mbox{follow by MP from some $\phi_{j}$, $\phi_{m}$, where $j<i$, $m<i$, and $\phi_{m}=\phi_{j}\rightarrow\phi_{i}$}\\
	\end{array}
	\end{center}\vspace{1em}
Showing that $\sum\vdash(\alpha\rightarrow\phi_{i})$ (d), (e), and (f) is done similar to (a), (b), and (c) above. All that is left, is to show $\sum\vdash(\alpha\rightarrow\phi_{i})$ for case (g).\vspace{1em}
	\renewcommand{\arraystretch}{1.5}
	\begin{center}
	\hspace{0em}
	\begin{array}[b]{l@{\hspace{.75cm}}l@{\hspace{1cm}:\,}l}
	(d1) & \sum\vdash(\alpha\rightarrow\phi_{i})\mbox{ \textbf{for cases d, e, f}} & \mbox{Similar to a, b, c}\\
	(g1) & \sum\vdash(\alpha\rightarrow\phi_{j}) & \mbox{Inductive Hyp.}\\
	(g2) & \sum\vdash(\alpha\rightarrow\phi_{m}) & \mbox{Inductive Hyp.}\\
	(g3) & \sum\vdash(\alpha\rightarrow(\phi_{j}\rightarrow\phi_{i})) & \mbox{Substitution, g1}\\
	(g4) & \sum\vdash((\alpha\rightarrow(\phi_{j}\rightarrow\phi_{i}))\rightarrow((\alpha\rightarrow\phi_{j})\rightarrow(\alpha\rightarrow\phi_{i}))) & \mbox{PC Axiom 2}\\
	(g5) & \sum\vdash((\alpha\rightarrow\phi_{j}\rightarrow(\alpha\rightarrow\phi_{i})) & \mbox{MP, g3, g4}\\
	(g6) & \sum\vdash(\alpha\rightarrow\phi_{i})\mbox{ \textbf{for case g}} & \mbox{MP, g5, g1}\\
	\end{array}
	\end{center}\vspace{1em}
This concludes the inductive step, which shows $\sum\vdash(\alpha\rightarrow\phi_{i})$ for all $i>1$, while the ``base'' case handles $i=1$. Letting $i=n$, we get $\sum\vdash(\alpha\rightarrow\phi_{n})$, which by substitution results in $\sum\vdash(\alpha\rightarrow\phi)$.\vspace{1em}
\Large
\begin{center}
$\therefore$  $\sum\cup\{\alpha\}\vdash\phi\Rightarrow\sum\vdash(\alpha\rightarrow\phi)$
\end{center} 
\end{document}
