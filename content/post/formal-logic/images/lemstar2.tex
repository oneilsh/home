\documentclass[12pt,letterpaper]{article}
\usepackage{graphicx, amssymb, amsmath, amsthm}

\setlength{\parindent}{2.5em}
\setlength{\oddsidemargin}{-.25in}
\setlength{\evensidemargin}{-.25in}
\setlength{\footskip}{.5in}
\setlength{\textwidth}{7in}
\setlength{\textheight}{9.5in}
%\setlength{\headheight}{0in} defaults to 0
%\setlength{\topskip}{0in} defualts to 0
%setlength{\headsep}{0in} defaults to 0
\setlength{\topmargin}{0in}

\newtheorem{defaultThm}{Theorum}
\theoremstyle{definition}
\newtheorem{definition}{Definition}
\newtheoremstyle{mythm}{1em}{1em}{}{1in}{}{}{.5em}{}
\newtheorem*{mythm}{Proof}

\begin{document}

\begin{center}
Shawn O'Neil\\
Math Logic\\
Fall 2004
\end{center}

\section*{\begin{center}Proof: ($\sum$ consistent $\rightarrow$ $\sum$ satisfiable) $\rightarrow$ ($\sum\models\alpha$ $\rightarrow$ $\sum\vdash\alpha$)\end{center}}
%\renewcommand{\baselinestretch}{2}
%\normalsize
%\hspace{\parindent}
This Lemma is used in the final steps of the Completeness theorem, which claims the second part of the implication: $\sum\models\alpha$ $\rightarrow$ $\sum\vdash\alpha$. If every model of $\sum$ is a model of $\alpha$, then there is a proof of $\alpha$ from $\sum$.
\vspace{1em}
\begin{list}{$\bullet$}{\setlength{\leftmargin}{2in}}
\item{\begin{math}\sum consistent \leftrightarrow \forall \beta(\sum\not\vdash(\beta\wedge\neg\beta))\end{math}}
\item{\begin{math}\sum satisfiable \leftrightarrow \exists A(A\models\sum)\end{math}}
\item{\begin{math}\sum\models\alpha \leftrightarrow \forall A(A\models\sum \rightarrow A\models\alpha)\end{math}}
\end{list}
\vspace{1em}
Now, using a proof by contradiction, we prove the lemma at hand. For a proof by contradiction in this example, we assume (a) $\sum$ consistent $\rightarrow$ $\sum$ satisfiable, (b) $\sum\models\alpha$, and (c) $\sum\not\vdash\alpha$.\vspace{1em}

\renewcommand{\arraystretch}{1.5}
\begin{center}
\hspace{0em}
\begin{array}[b]{l@{\hspace{.75cm}}l@{\hspace{1cm}:\,}l}
(1) & (\forall \beta(\sum\not\vdash(\beta\wedge\neg\beta))) \rightarrow (\exists A(A\models\sum)) & (a)\\
(2) & \forall A(A\models\sum \rightarrow A \models\alpha) & (b)\\
(3) & \sum\not\vdash\alpha & (c)\\
(4) & \forall \beta(\sum\not\vdash(\beta\wedge\neg\beta)) & (3) tricky...\\
(5) &\forall \beta(\sum\bigcup\{\neg\alpha\}\not\vdash(\beta\wedge\neg\beta)) & (3),(4)\\
(6) & \exists A(A\models\sum\bigcup\{\neg\alpha\}) & (1),(5)\\
(7) & \exists A(A\models\neg\alpha\wedge A\models\sum) & (6)\\
(8) & \exists A(A\models\neg\alpha\wedge A\models\alpha) & (2),(7)
\end{array}
\end{center}

However, the last step contradicts the definition of a model, thus one of the initial assumptions must be false, namely the third one which states $\sum\not\vdash\alpha$.
\vspace{1em}
\Large
\begin{center}
$\therefore$ ($\sum$ consistent $\rightarrow$ $\sum$ satisfiable) $\rightarrow$ ($\sum\models\alpha$ $\rightarrow$ $\sum\vdash\alpha$) 
\end{center} 
\end{document}
